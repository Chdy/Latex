\subsection{时间客户端及服务器端}
\subsubsection{客户端}
\begin{minted}{c}
# include </Users/dengyan/ClionProjects/Linux/linux.h>

int main(int argc, char *argv[]) {
    int sockfd, n;
    char buf[4097];
    struct sockaddr_in servaddr;

    if ((sockfd = socket(AF_INET, SOCK_STREAM, 0)) == -1)
        err_sys("socket error");

    bzero(&servaddr, sizeof(servaddr));
    servaddr.sin_family = AF_INET;
    servaddr.sin_port = htons(13333);
    if (inet_pton(AF_INET, "127.0.0.1", &servaddr.sin_addr) == -1)
        err_quit("inet_pton error");

    if (connect(sockfd, (struct sockaddr *)&servaddr, sizeof(servaddr)) < 0)
        err_sys("connect error");

    while ((n = read(sockfd, buf, 4096)) > 0) {
        write(STDOUT_FILENO, buf, n);
    }
    return 0;
}
\end{minted}
\subsubsection{服务器端}
\begin{minted}{c}
# include </Users/dengyan/ClionProjects/Linux/linux.h>

int main(int argc, char *argv[]) {
    int sockfd, n, fd;
    struct sockaddr_in servaddr;

    if ((sockfd = socket(AF_INET, SOCK_STREAM, 0)) < 0)
        exit(-1);

    bzero(&servaddr, sizeof(servaddr));
    servaddr.sin_port = htons(13333);
    servaddr.sin_family = AF_INET;
    if (inet_pton(AF_INET, "127.0.0.1", &servaddr.sin_addr) <= 0)
        exit(-1);

    if (bind(sockfd, (struct sockaddr *)&servaddr, sizeof(servaddr)) != 0) {
        exit(-1);
    }

    listen(sockfd, 5);

    while (1) {
        if ((fd = accept(sockfd, NULL, NULL)) < 0) {
            printf("%s", strerror(errno));
            continue;
        }
        time_t t = time(NULL);
        char * str = ctime(&t);
        write(fd, str, strlen(str));
        close(fd);
    }

}
\end{minted}
\newpage

\subsection{使用select实现的echo服务器(TCP及UDP同时支持)}
\subsubsection{客户端}
\begin{minted}{c}
# include </Users/dengyan/ClionProjects/Linux/linux.h>

int main(int argc, char *argv[]) {
    char buf[4096];
    int tcpfd, udpfd, len, choose, n;
    struct sockaddr_in servaddr, cliaddr;

    if ((tcpfd = socket(AF_INET, SOCK_STREAM, 0)) < 0)
        err_sys("tcp socket error");

    bzero(&servaddr, sizeof(servaddr));
    servaddr.sin_family = AF_INET;
    servaddr.sin_port = htons(23333);
    servaddr.sin_addr.s_addr = htonl(INADDR_LOOPBACK);

    if ((udpfd = socket(AF_INET, SOCK_DGRAM, 0)) < 0)
        err_sys("udp socket error");

    if (connect(tcpfd, (struct sockaddr *)&servaddr, sizeof(servaddr)) < 0)
        err_sys("connect error");


    connect(udpfd, (struct sockaddr *)&servaddr, sizeof(servaddr));

    for (;;) {
        printf("1: TCP echo\n");
        printf("2: UDP echo\n");

        scanf("%d", &choose);
        switch (choose) {
            case 1:
                n = read(STDIN_FILENO, buf, 4096);
                write(tcpfd, buf, n);
                n = read(tcpfd, buf, n);
                write(STDOUT_FILENO, buf, n);
                break;
            case 2:
                n = read(STDIN_FILENO, buf, 4096);
                write(udpfd, buf, n);
                n = read(udpfd, buf, n);
                write(STDOUT_FILENO, buf, n);
                break;
            default:
                break;
        }
    }

    return 0;
}
\end{minted}

\subsubsection{服务器端}
\begin{minted}{c}
# include </Users/dengyan/ClionProjects/Linux/linux.h>

void handler(int s)
{
    pid_t pid;
    while ((pid = waitpid(-1, NULL, WNOHANG)) > 0)
        printf("child pid %d terminate ", pid);
}

int main(int argc, char *argv[]) {
    int listenfd, clifd, udpfd, n;
    socklen_t len;
    int on = 1, maxfd = -1;
    struct sockaddr_in servaddr, cliaddr;

    if ((listenfd = socket(AF_INET, SOCK_STREAM, 0)) < 0)
        err_quit("socket error");

    bzero(&servaddr, sizeof(servaddr));
    servaddr.sin_family = AF_INET;
    servaddr.sin_port = htons(23333);
    servaddr.sin_addr.s_addr = htonl(INADDR_ANY);

    signal(SIGCHLD, handler);
    
    if (setsockopt(listenfd, SOL_SOCKET, SO_REUSEADDR, &on, sizeof(on)) == -1)
        err_quit("setsockopt error");
    
    if (bind(listenfd, (struct sockaddr *)&servaddr, sizeof(servaddr)) < 0)
        err_quit("bind error");

    listen(listenfd, 5);

    if ((udpfd = socket(AF_INET, SOCK_DGRAM, 0)) < 0)
        err_quit("udp socket error");

    if (bind(udpfd, (struct sockaddr *)&servaddr, sizeof(servaddr)) < 0)
        err_quit("bind error");

    maxfd = listenfd > udpfd?listenfd:udpfd;
    fd_set set;
    FD_ZERO(&set);

    for ( ;; ) {
        FD_SET(listenfd, &set);
        FD_SET(udpfd, &set);

        if (select(maxfd + 1, &set, NULL, NULL, NULL) < 0) {
            if (errno == EINTR)
                continue;
            else
                err_sys("select error");
        }

        if (FD_ISSET(listenfd, &set)) {
            char name[40];
            len = sizeof(cliaddr);
            clifd = accept(listenfd, (struct sockaddr *)&cliaddr, &len);
            printf("connect from %s:%d\n", inet_ntop(AF_INET, &cliaddr.sin_addr, name, sizeof(name)), ntohs(cliaddr.sin_port));
            fflush(stdout);

            if (fork() == 0) {
                char buf[4096];
                close(listenfd);
                while ((n = read(clifd, buf, sizeof(buf))) > 0) {
                    write(clifd, buf, n);
                }
                exit(-1);
            }
            close(clifd);
        }

        if (FD_ISSET(udpfd, &set)) {
            char buf[4096];
            len = sizeof(cliaddr);
            n = recvfrom(udpfd, buf, sizeof(buf), 0, (struct sockaddr *)&cliaddr, &len);
            sendto(udpfd, buf, n, 0, (struct sockaddr *)&cliaddr, len);
        }
    }
}
\end{minted}
\newpage

\subsection{使用getaddrinfo基于tcp实现的时间服务器}
\subsubsection{客户端}
\begin{minted}{c}
# include </Users/dengyan/ClionProjects/Linux/linux.h>

int main(int argc, char *argv[]) {
    int sockfd, n;
    char buf[100];
    char recvline[MAXLINE + 1];
    socklen_t len;
    struct sockaddr_storage ss;

    sockfd = tcp_connect("localhost", "13333");
    len = sizeof(ss);
    getpeername(sockfd, (struct sockaddr *)&ss, &len);
    printf("connected to %s\n", len==16?inet_ntop(AF_INET, &((struct sockaddr_in *)&ss)->sin_addr, buf, 100):inet_ntop(AF_INET6, &((struct sockaddr_in *)&ss)->sin_addr, buf, 100));

    while ((n = read(sockfd, recvline, MAXLINE)) > 0) {
        write(STDOUT_FILENO, recvline, n);
    }
    return 0;
}
\end{minted}

\subsubsection{服务器端}
\begin{minted}{c}
# include </Users/dengyan/ClionProjects/Linux/linux.h>

int main(int argc, char *argv[]) {
    int listenfd, connfd;
    socklen_t len;
    char buf[MAXLINE];
    time_t t;
    struct sockaddr_storage cliaddr;

    listenfd = tcp_listen(NULL, "13333", NULL);

    for ( ;; ) {
        len = sizeof(cliaddr);
        connfd = accept(listenfd, (struct sockaddr *)&cliaddr, &len);
        printf("connected to %s\n", len==16?inet_ntop(AF_INET, &((struct sockaddr_in *)&cliaddr)->sin_addr, buf, 100):inet_ntop(AF_INET6, &((struct sockaddr_in *)&cliaddr)->sin_addr, buf, 100));

        t = time(NULL);
        char * s = ctime(&t);
        write(connfd, s, strlen(s));
        close(connfd);
    }
    return 0;
}
\end{minted}
\newpage

\subsection{观察TCP阻塞窗口}
\subsubsection{客户端}
\begin{minted}{c}
# include </Users/dengyan/ClionProjects/Linux/linux.h>

int main(int argc, char *argv[]) {
    int fd, n;
    char buf[MAXLINE];
    struct sockaddr_in servaddr;
    struct tcp_connection_info tcpinfo;
    int len = sizeof(tcpinfo);
    fd = socket(AF_INET, SOCK_STREAM, 0);

    int sfd = open("/dev/zero", O_RDONLY);
    read(sfd, buf, 1000);

    bzero(&servaddr, sizeof(servaddr));
    servaddr.sin_port = htons(23333);
    servaddr.sin_family = AF_INET;
    servaddr.sin_addr.s_addr = htonl(INADDR_LOOPBACK);

    connect(fd, &servaddr, sizeof(servaddr));
    printf("接收窗口 发送窗口 发送阻塞窗口\n");
    while (write(fd, buf, 1000) > 0)
    {
        getsockopt(fd, IPPROTO_TCP, TCP_CONNECTION_INFO, &tcpinfo, &len);
        printf("%d %d %d\n", tcpinfo.tcpi_rcv_wnd, tcpinfo.tcpi_snd_wnd, tcpinfo.tcpi_snd_cwnd);
        sleep(0.2);
    }
}
\end{minted}

\subsubsection{服务器端}
\begin{minted}{c}
# include </Users/dengyan/ClionProjects/Linux/linux.h>

int main(int argc, char *argv[]) {
    setbuf(stdout, NULL);
    int listenfd, n = 4;
    int rcv = 0;
    char buf[MAXLINE];
    struct tcp_connection_info tcpinfo;
    int len = sizeof(tcpinfo);

    struct addrinfo hints, *res, *ressave;

    bzero(&hints, sizeof(hints));
    hints.ai_family = AF_INET;
    hints.ai_socktype = SOCK_STREAM;
    hints.ai_flags = AI_PASSIVE|AI_NUMERICSERV;

    if (getaddrinfo(NULL, "23333", &hints, &res) != 0)
        err_quit("getaddrinfo error");

    ressave = res;
    do {
        listenfd = socket(AF_INET, SOCK_STREAM, 0);

        setsockopt(listenfd, SOL_SOCKET, SO_REUSEADDR, &n, sizeof(int));

        if (bind(listenfd, res->ai_addr, res->ai_addrlen) == 0)
            break;

        close(listenfd);
    } while ((res = res->ai_next) != NULL);

    if (res == NULL)
        err_quit("don't find the socket to bind");

    freeaddrinfo(ressave);

    getsockopt(listenfd, SOL_SOCKET, SO_RCVBUF, &rcv, &n);
    printf("rcvbuf = %d\n", rcv);
    rcv = 262000;
    setsockopt(listenfd, SOL_SOCKET, SO_RCVBUF, &rcv, n);
    printf("rcvbuf = %d\n", rcv);
    listen(listenfd, 5);

    int fd = accept(listenfd, NULL, 0);
    printf("接收窗口 发送窗口 发送阻塞窗口\n");

    for (;;) {
        getsockopt(fd, IPPROTO_TCP, TCP_CONNECTION_INFO, &tcpinfo, &len);
        printf("%d %d %d\n", tcpinfo.tcpi_rcv_wnd, tcpinfo.tcpi_snd_wnd, tcpinfo.tcpi_snd_cwnd);
        sleep(1);
    }
    return 0;
}
\end{minted}

\subsection{XX}
\subsubsection{客户端}
\begin{minted}{c}

\end{minted}

\subsubsection{服务器端}
\begin{minted}{c}

\end{minted}


