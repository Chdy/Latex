\subsection{netinet/in.h}
\subsubsection{in\li addr}
\begin{minted}{c}
struct in_addr {
    in_addr_t s_addr; //IPv4地址,通常为uint32_t
};
/*
 * 需要注意的是,对32位IPv4地址进行访问时有两种不同的访问方法
 * 一种是访问结构体,另一种是访问结构体中的成员
 * 在当中参数时需要小心处理,因为编译器处理结构体和
 * 无符号整形采用不同的方式,结构体作为左值时无法随意赋值
 * 所以如果想直接对地址赋值需要访问结构体中的成员
 */
\end{minted}

\subsubsection{sockaddr\li in}
\begin{minted}{c}
struct sockaddr_in {
    __uint8_t   sin_len; //通常不需要设置和检查
    sa_family_t sin_family; //地址族(AF_INET),通常为8位的无符号整形
    in_port_t   sin_port; //TCP或UDP端口,一般为uint16_t
    struct in_addr sin_addr; //IPv4地址,通常为uint32_t
    char        sin_zero[8]; //用于填充,达到最小16个字节
};
\end{minted}

\subsubsection{in6\li addr}
\begin{minted}{c}
//此结构在mac上为<netinet6/in6.h>中
typedef struct in6_addr {
    union {
        __uint8_t   __u6_addr8[16];
        __uint16_t  __u6_addr16[8];
        __uint32_t  __u6_addr32[4];
    } __u6_addr; /* 128-bit IP6 address */
} in6_addr_t;
//通配地址赋值给此结构体可以使用in6_addr = in6addr_any
\end{minted}

\subsubsection{sockaddr\li in6}
\begin{minted}{c}
//此结构在mac上为<netinet6/in6.h>中
struct sockaddr_in6 {
    __uint8_t   sin6_len; //通常不需要设置和检查
    sa_family_t sin6_family; //地址族(AF_INET6)
    in_port_t   sin6_port;
    __uint32_t  sin6_flowinfo;
    struct in6_addr	sin6_addr;
    __uint32_t  sin6_scope_id;
};
\end{minted}

\subsection{sys/socket.h}
\subsubsection{sockaddr}
\begin{minted}{c}
//16个字节,不足以容纳容纳IPv6结构
struct sockaddr {
    __uint8_t    sa_len;
    sa_family_t sa_family;
    char        sa_data[14];
};
\end{minted}

\subsubsection{sockaddr\li storage}
\begin{minted}{c}
/*
 * 由于sockaddr不足以容纳所有的地址结构,所以新的通用套接字结构产生了
 * 它足以容纳所有类型的套接字地址结构,而且满足最苛刻的对齐要求
 * 但是需要将其转换为特定的类型后,才可以访问其内部字段
 */
struct sockaddr_storage {
    __uint8_t   ss_len;
    sa_family_t ss_family;
    char        __ss_pad1[_SS_PAD1SIZE];
    __int64_t   __ss_align;
    char        __ss_pad2[_SS_PAD2SIZE];
};
\end{minted}

\subsubsection{iovec}
\begin{minted}{c}
struct iovec {
    void  *iov_base;    /* 缓冲区地址 */
    size_t iov_len;     /* 缓冲区有效数据长度 */
};
\end{minted}

\subsubsection{msghdr}
\begin{minted}{c}
struct msghdr {
    void         *msg_name;       /* 地址 */
    socklen_t     msg_namelen;    /* 地址长度 */
    struct iovec *msg_iov;        /* 用于散步读聚集写的数组 */
    size_t        msg_iovlen;     /* 数组长度 */
    void         *msg_control;    /* 指向控制信息头 */
    size_t        msg_controllen; /* 控制信息大小 */
    int           msg_flags;      /* 标志 */
};
\end{minted}

\subsection{netdb.h}
\subsubsection{hostent}
\begin{minted}{c}
struct hostent {
    char  *h_name;            /* 正式主机名 */
    char **h_aliases;         /* 别名列表 */
    int    h_addrtype;        /* 主机地址类型: AF_INET */
    int    h_length;          /* 地址长度: 4 */
    char **h_addr_list;       /* 其中的每个指针指向in_addr结构体 */
}
\end{minted}

\subsubsection{servent}
\begin{minted}{c}
struct servent {
    char  *s_name;       /* 正式服务名 */
    char **s_aliases;    /* 别名列表 */
    int    s_port;       /* 端口号(网络序) */
    char  *s_proto;      /* 使用的协议名 */
}
\end{minted}

\subsubsection{addrinfo}
\begin{minted}{c}
struct addrinfo {
    int              ai_flags; /* 控制返回信息 */
    int              ai_family; /* AF_INET|AF_INET6|AF_UNSPEC 意义为获取哪种地址结构 */
    int              ai_socktype; /* SOCK_XXX */
    int              ai_protocol; /* 一般为0 */
    socklen_t        ai_addrlen; /* ai_addr长度 */
    struct sockaddr *ai_addr; /* 存放地址信息 */
    char            *ai_canonname;
    struct addrinfo *ai_next; /* 指向链表上的下一个addrinfo */
};

/* ------------ai_flags------------
 * AI_PASSIVE 套接字将用于被动打开(服务器端)
 * AI_CANONNAME 告知getaddrinfo返回主机的规范名字
 * AI_V4MAPPED 如果ai_family指定的是IPv6,那么如果没有可用的AAAA记录,则返回一条IPv4映射的IPv6地址)
 * AI_ADDRCONFIG表示按照主机配置配置选择返回的地址类型
 * AI_ALL表示查找IPv4和IPv6(IPv6需要指定AI_V4MAPPED后还会返回IPv4映射的IPv6地址)
 * AI_NUMERICHOST表示以数字格式指定主机地址
 * AI_NUMERICSERV表示以数字形式(端口号)指定服务
 */
\end{minted}
