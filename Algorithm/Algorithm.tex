\documentclass[12pt,a3paper]{article}
\usepackage{ctex}
\usepackage{minted}
\usepackage{fancyhdr}
\usepackage[left=1.2in,right=1.2in,top=1in,bottom=0.9in]{geometry}
%\usepackage[total={6.5in,8.75in},
%top=1.2in, left=0.9in, includefoot]{geometry}
\usepackage{lastpage}
\usepackage{color}
\usepackage{setspace} %行间距
\usepackage{lmodern} %下行的使用需要这行
\usepackage[T1]{fontenc} %用于处理下划线等输出
\usepackage{textcomp} %\textasciitilde中划线 \textunderscore下划线
\usepackage{chngpage} %使用adjustwidth环境偏移整个块 
\usepackage{cite} %引用bib中的文献

\usepackage[pdfstartview=FitH,
CJKbookmarks=true,
bookmarksnumbered=true,
bookmarksopen=true,
colorlinks,
pdfborder=001,
linkcolor=blue,
anchorcolor=blue,
citecolor=blue,
]{hyperref} %实现目录和章节的超链接
\hypersetup{hidelinks}

\pagestyle{fancy}
\fancyhf{}
\rfoot{\thepage}
\rhead{\rightmark}
\renewcommand{\headrulewidth}{0pt}
%页眉横线宽度

%\setboolean{@twoside}{true} %从oneside转换为twoside,这样才能有单数页和双数页码


\begin{document}
\title{算法笔记}
\author{邓岩\footnote{837123564@qq.com}}
\date{2018 年 11 月 13 日}
\maketitle
\thispagestyle{empty}
\newpage
\tableofcontents
\newpage

\setminted[c++,c,python]{
			   breaklines,
			   %breakbefore= ,
			   %breakanywhere,
               %linenos, %行号
               numbersep=10pt, %行号与代码距离
               autogobble, %去掉前两列
               %frame=lines, %边缘线
               framesep=5pt, %代码与框架距离
               %以下选项需要加在环境选项中才有效
               %xleftmargin=2em,
			   %xrightmargin=0em,
			   %mathescape,
}

\section{\color[rgb]{0.2,0.4,0.6}{第一章}}
\subsection{No.1}
\begin{adjustwidth}{2em}{0em}
\begin{minted}[linenos,mathescape]{c++}
// $x_a$
\end{minted}
\end{adjustwidth}

\section{\color[rgb]{0.2,0.4,0.6}{动态规划}}
\subsection{切钢条}
\begin{minted}[linenos,xleftmargin=2em,
			   xrightmargin=0em,]{c}
//
// Created by 邓岩 on 2018/11/3.
//

# include <stdio.h>
# include <stdlib.h>
# include <string.h>

//动态规划切钢条

void bottom_to_up(int n) //自底向上法
{
    int rr[11] = {0, 1, 5, 8, 9, 10, 17, 17, 20, 24, 30};
    int * r = (int *)calloc(n + 1, sizeof(int));
    memcpy(r, rr, sizeof(int) * 11);
    int * s = (int *)calloc(n + 1, sizeof(int));

    for (int i = 1; i < n+1; ++i) {
        for (int j = 0; j <= i; ++j) {
            if (r[i] < r[j] + r[i - j]) {
                r[i] = r[j] + r[i - j];
                s[i] = j;
            }
        }
    }
    printf("  长度: ");
    for (int k = 0; k < n + 1; ++k) {
        printf("%3d ", k);
    }
    printf("\n  价格: ");
    for (int k = 0; k < n + 1; ++k) {
        printf("%3d ", r[k]);
    }
    printf("\n  切割: ");
    for (int k = 0; k < n + 1; ++k) {
        printf("%3d ", s[k]);
    }
}

int up_to_bottom(int n, int * rr, int * r) //带备忘的自顶向下法
{
    if (r[n] > 0)
        return r[n];
    for (int i = 1; i < n + 1; ++i) {
        if (r[n] < rr[i] + up_to_bottom(n - i, rr, r)) {
            r[n] = rr[i] + up_to_bottom(n - i, rr, r);
        }
    }
    return r[n];
}

int main(void)
{
    int n ;
    printf("请输入钢条长度: ");
    scanf("%d", &n);
    int t[11] = {0, 1, 5, 8, 9, 10, 17, 17, 20, 24, 30};
    int * rr = (int *)calloc(n + 1, sizeof(int));
    memcpy(rr, t, sizeof(int) * 11);
    int * r = (int *)calloc(n + 1, sizeof(int));
    printf("%d", up_to_bottom(n, rr, r));
}
\end{minted}

\subsection{矩阵链乘法}
\begin{minted}[linenos,xleftmargin=2em,
			   xrightmargin=0em,]{c}
//
// Created by 邓岩 on 2018/11/3.
//

# include <stdio.h>
# include <stdlib.h>
# include <string.h>

//动态规划处理矩阵链乘法


void print_optimal_parens(int (*s)[7], int i, int j) //这里的7为二维数组s的列数
{
    if (i == j)
        printf("A%d", i);
    else {
        printf("(");
        print_optimal_parens(s, i, s[i][j]);
        print_optimal_parens(s, s[i][j] + 1, j);
        printf(")");
    }
}

void matrix_chain_order(const int * p, int n)
{
    int j;
    int (*m)[n + 1] = calloc((n + 1) * (n + 1), sizeof(int));
    int (*s)[n + 1] = calloc((n + 1) * (n + 1), sizeof(int));
    for (int l = 2; l <= n; ++l) {
        for (int i = 1; i <= n - l + 1; ++i) {
            j = i + l - 1;
            m[i][j] = INT32_MAX;
            for (int k = i; k < j; ++k) {
                int q = m[i][k] + m[k + 1][j] + p[i - 1] * p[k] * p[j];
                if (q < m[i][j]) {
                    m[i][j] = q;
                    s[i][j] = k;
                }
            }
        }
    }
    for (int i1 = 0; i1 < n + 1; ++i1) {
        for (int i = 0; i < n + 1; ++i) {
            printf("%5d ", m[i1][i]);
        }
        for (int i = 0; i < n + 1; ++i) {
            printf("%5d ", s[i1][i]);
        }
        printf("\n");
    }
    print_optimal_parens(s, 1, n);
    free(m);
    free(s);
}

int main(void)
{
    int p[] = {30, 35, 15, 5, 10, 20, 25};
    //有六个矩阵,p[0]存放着第一个矩阵的行,其余的p[i]表示第i个矩阵的列数
    matrix_chain_order(p, 6);
}
\end{minted}

\subsection{最长公共子序列}
\begin{minted}[linenos,xleftmargin=2em,
			   xrightmargin=0em,
]{c}
//
// Created by 邓岩 on 2018/11/3.
//

# include <stdio.h>
# include <stdlib.h>
# include <string.h>
# include <unistd.h>
# include <sys/stat.h>
# include <fcntl.h>

void LCS_PRINT(char ** t, char * y, int i, int j)
{
    int n = strlen(y);
    char (*b)[n + 1] = t;
    if (i == 0 || j == 0)
        return;
    if (b[i][j] == '\\') {
        LCS_PRINT(t, y, i - 1, j - 1);
        printf("%c", y[j - 1]);
    } else if (b[i][j] == '|') {
        LCS_PRINT(t, y, i - 1, j);
    } else {
        LCS_PRINT(t, y, i, j - 1);
    }
}

void LCS_LENGTH(char * x, char * y)
{
    int len1 = strlen(x);
    int len2 = strlen(y);
    char (*b)[len2 + 1] = calloc((len1 + 1) * (len2 + 1), sizeof(char));
    int (*c)[len2 + 1] = calloc((len1 + 1) * (len2 + 1), sizeof(int));
    for (int i = 1; i <= len1; ++i) {
        for (int j = 1; j <= len2; ++j) {
            if (*(x + i - 1) == *(y + j - 1)) {
                c[i][j] = c[i - 1][j - 1] + 1;
                b[i][j] = '\\';
            } else if (c[i - 1][j] >= c[i][j - 1]) { 
            //c[i][j],c[i - 1][j],c[i][j - 1]是相邻的,也就是说箭头指向有较大最大LCS的方向
                c[i][j] = c[i - 1][j];
                b[i][j] = '|';
            } else {
                c[i][j] = c[i][j - 1];
                b[i][j] = '-';
            }
        }
    }
    /*for (int i = 0; i <= len1; ++i) {
        for (int j = 0; j <= len2; ++j) {
            printf("%d ", c[i][j]);
        }
        for (int j = 0; j <= len2; ++j) {
            printf("%c ", b[i][j]);
        }
        printf("\n");
    }*/
    LCS_PRINT(b, y, len1, len2);
}

int main(void)
{
    char * file1 = "/Users/dengyan/exam.c";
    char * file2 = "/Users/dengyan/exam";
    struct stat stat1;
    stat(file1, &stat1);
    char * x = (char *)malloc(stat1.st_size + 1);
    int fdx = open(file1, O_RDONLY);
    read(fdx, x, stat1.st_size);
    x[stat1.st_size] = 0;
    stat(file2, &stat1);
    char * y = (char *)malloc(stat1.st_size + 1);
    int fdy = open(file2, O_RDONLY);
    read(fdy, y, stat1.st_size);
    y[stat1.st_size] = 0;

    //char x[100] = "ABCBDAB";
    //char y[100] = "BDCABA";
    LCS_LENGTH(x, y);
}
\end{minted}
\input{section/tree}

\newpage
\bibliographystyle{unsrt}
\bibliography{}


\end{document}