\documentclass[12pt,a4paper]{article}
%\usepackage{ctex}
\usepackage{CJK}
\usepackage{color}
\usepackage[left=1.2in,right=1.2in,top=1.5in,bottom=1.5in]{geometry} %导入页面边距宏
\usepackage{fancyhdr} %导入页眉页脚宏
\usepackage{lastpage} %引用页数信息
\usepackage{setspace} %行间距
\usepackage{graphicx} %插入图片
\usepackage{amsmath} %数学公式处理 align环境
\usepackage{tikz} %画图
\usepackage{pgfplots} %画图用 axis环境
\usepackage[T1]{fontenc} %用于处理下划线等输出
\usepackage{textcomp} %\textasciitilde中划线 \textunderscore下划线
\usepackage[pdfstartview=FitH,
CJKbookmarks=true,
bookmarksnumbered=true,
bookmarksopen=true,
colorlinks,
pdfborder=001,
linkcolor=blue,
anchorcolor=blue,
citecolor=blue,
]{hyperref} %实现目录和章节的超链接
\hypersetup{hidelinks}

\pagestyle{fancy} %页面样式
\lhead{Chdy} %左页眉
%\chead{\begin{CJK}{UTF8}{gbsn}{\leftmark} \end{CJK}} %中间页眉
\rhead{\begin{CJK}{UTF8}{gbsn}{\leftmark} \end{CJK}} %右页眉
\rfoot{\thepage} 
\cfoot{}

\begin{document}
\begin{CJK}{UTF8}{gbsn}

\author{Dy\footnote{837123564@qq.com}}
\title{Latex}
\date{\today}
\maketitle
\thispagestyle{empty} %去除因maketitle而产生的页码
\newpage
\tableofcontents

\newpage
MTU表示最大传输单元,由硬件规定,如以太网的MTU为1500字节。

加上链路层的头部和FCS可以达到1518,由于有效链路层的数据帧最小需要64字节,所以ip数据报最小为46字节,MSS表示最大段大小

MSS:maximum segment size,最大分节大小,为TCP数据包每次传输的最大数据分段大小,一般由发送端向对端TCP通知对端在每个分节中能发送的最大TCP数据。MSS值为MTU值减去IPv4 Header(20 Byte)和TCP header(20 Byte)得到。

TCP中的头部长度字段的单位是4字节

IP中的偏移量字段单位为8字节

IP中的头部长度字段单位是4字节

在建立连接时会告诉对方自己的接收窗口的大小,接收方收到信息后可以设置小于等于该值的发送窗口,并回应给发送方自己的接收窗口,使发送方设置合适的发送窗口

当客户端发送syn数据包,会变为SYN\_SENT状态,当服务器接受到syn数据包后,返回一个syn数据包,此时服务器变为SYN\_RCVD状态,当客户端接受到服务器端发送的syn数据包后,会发送一个确认包给服务器端,此时客户端变为ESTABLISHED状态,当服务器接受到这个确认包后,服务器会变成ESTABLISHED状态

想观察SYN\_SENT状态可以去尝试与一个不存在的ip地址建立连接

想观察SYN\_RCVD状态可以去接受到一个不存在的ip地址发送过来的SYN请求

\end{CJK}
\end{document}